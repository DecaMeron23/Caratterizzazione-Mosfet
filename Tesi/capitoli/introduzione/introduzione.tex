Uno degli elementi principali delle componenti elettroniche è il transistore, ovvero un dispositivo a stato solido composto da materiali semiconduttori che sfrutta le proprietà fisiche della giunzione P-N per assumere il ruolo di amplificatore, di interruttore oppure di resistenza variabile.
Esso è impiegato in svariati campi, dall'automazione industriale all'aviazione, dalle telecomunicazioni fino all'ambito medico.
Esistono diversi tipi di transistore che si differenziano per la loro struttura e, di conseguenza, per i loro funzionanamento e utilizzo. I principali sono:
\begin{itemize}
    \item \textit{Bipolar Junction Transistor} (BJT),
    \item \textit{Junction Field-Effect Transistor} (JFET),
    \item \textit{Metal-Oxide-Semiconductor Field-Effect Transistor} (MOSFET).
\end{itemize}
Il dispositivo più comune è il MOSFET. Esso è la componente principale del CMOS (\textit{Complementary metal-oxide-semiconductor}), costituito da un MOSFET a canale N e uno a canale P. Uno fatttore critico di questo dispositivo è la sua dimensione. Una delle frontiere di ricerca nei confronti del CMOS è la sua miniaturizzazione. Infatti, ridurre le dimensioni di tale dispositico comporta diversi vantaggi, come: riduzione della potenza dissipata, magiore resistenza agli effetti delle radiazioni e aumento della densita di transistor per unità d'area.  

\vspace*{0.5cm}

L'obiettivo di questo lavoro di tesi, realizzato presso il laboratorio di microelettronica dell'\textit{Università degli studi di Bergamo}, è quello di osservare gli effetti delle radiazioni ionizzanti sui parametri statici (ad esempio: tensione di soglia e correnti di perdita $I_{OFF}$) di CMOS in teconologia $28nm$. Inoltre, si vuole esaminare se e come tali parametri possono migliore nel caso in cui i dispositivi non sono più soggetti a radiazioni.     

\vspace*{0.5cm}

Nel capitolo \ref{cap1} verranno introdotte le caratteristiche principali del transistore MOSFET.
Si darà una breve descrizione della struttura dei dispositivi analizzati (sia per i transistori a canale N che a canale P) e a seguire si introdurranno i diversi parametri statici e le regioni di funzionamento. Verranno presentati il modello di piccolo segnale e le diverse sorgenti di rumore cui il transistore è soggetto. Il capitolo \ref{cap1} termina con l'analisi di come le radiazioni possono influire sui parametri statici e come si possono mitigare queste variazioni.

Nel capitolo \ref{cap2}, oltre a descrivere i dispositivi analizzati e le procedure di estrazione dei dati grezzi, verranno presentati i principali parametri statici, con un'analisi approfondita della tensione di soglia ($V_{th}$). Per ognuno di essi, si descriverà come possono essere ricavati dai dati grezzi e si mostreranno i valori estratti e come variano all'aumentare della dose assorbita.   
