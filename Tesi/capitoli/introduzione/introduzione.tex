Uno degli elementi principali delle componenti elettroniche è il transistore, ovvero un dispositivo a stato solido composto da materiali semiconduttori che sfrutta le proprietà fisiche della giunzione P-N per assumere il ruolo di amplificatore, di interruttore oppure di resistenza variabile.
Esso è impiegato in svariati campi, dall'automazione industriale all'aviazione, dalle telecomunicazioni fino all'ambito medico.
Esistono diversi tipi di transistore che si differenziano per la loro struttura e, di conseguenza, per il loro funzionamento e utilizzo. I principali sono:
\begin{itemize}
    \item \textit{Bipolar Junction Transistor} (BJT),
    \item \textit{Junction Field-Effect Transistor} (JFET),
    \item \textit{Metal-Oxide-Semiconductor Field-Effect Transistor} (MOSFET).
\end{itemize}
Il dispositivo più diffuso oggigiorno è il MOSFET. che viene fabbricato in tecnologia CMOS (\textit{Complementary metal-oxide-semiconductor}), costituita da MOSFET a canale N e a canale P. Nel corso degli ultimi decenni, l'evoluzione della tecnologia CMOS ha consentito la miniaturizzazione dei dispositivi con conseguente riduzione dei consumi per singolo transistore e un forte incremento di dispositivi per unità di area.

\vspace*{0.5cm}

L'obiettivo di questo lavoro di tesi, realizzato presso il laboratorio di microelettronica dell'\textit{Università degli studi di Bergamo}, è quello di studiare gli effetti delle radiazioni ionizzanti sui parametri statici (ad esempio: variazione della tensione di soglia e della corrente di \emph{leakage} $I_{OFF}$) di transistori in tecnologia $28nm$. 

\vspace*{0.5cm}

Nel capitolo \ref{cap1} verranno introdotte le caratteristiche principali del transistore MOSFET.
Si darà una breve descrizione della struttura dei dispositivi analizzati (sia per i transistori a canale N che a canale P) e a seguire si introdurranno i parametri statici del transistore e le regioni di funzionamento. Verranno presentati il modello di piccolo segnale e le diverse sorgenti di rumore cui il transistore è soggetto. Il capitolo \ref{cap1} termina con l'analisi di come le radiazioni possono influire sui parametri statici e come si possono mitigare queste variazioni.

Nel capitolo \ref{cap2}, oltre a descrivere i dispositivi analizzati e le procedure di estrazione dei dati grezzi, verranno presentati i principali parametri statici, con un'analisi approfondita delle tecniche di estrazione della tensione di soglia ($V_{th}$) dalle caratteristiche statiche acquisite mediante la strumentazione di laboratorio. Si mostreranno inoltre le caratteristiche della transconduttanza di canale $g_m$, della corrente di \emph{leakage} $I_{off}$, della massima corrente di canale $I_{on}$ e del guadagno intrinseco in funzione della dose di radiazione ionizzante assorbita fino a $3 Grad$ e post \emph{annealing} a $100\degree C$ per $24$ ore.
