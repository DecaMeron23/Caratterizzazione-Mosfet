Uno degli elementi principali delle componenti elettroniche è sicuramente il transistore, ovvero un dispositivo che sfrutta le proprietà di materiali semiconduttori, come il silicio, per assumere il ruolo di amplificatore, interruttore oppure di una resistenza variabile.
Esso è impiegato in svariati campi, dall'automazione industriale all'aviazione, dalle telecomunicazioni fino in ambito medico.
I transistori possono variare a seconda della struttura, i principali sono:
\begin{itemize}
    \item \textit{Bipolar Junction Transistor} (BJT),
    \item \textit{Junction Field-Effect Transistor} (JFET),
    \item \textit{Metal-Oxide-Semiconductor Field-Effect Transistor} (MOSFET).
\end{itemize}
Il dispositivo più presente nel mercato è il MOSFET, esso è la componente principale del CMOS (\textit{Complementary metal-oxide-semiconductor}) costituito da un MOSFET a canale N e uno a canale P. La continua corsa per ridurre le dimensioni del CMOS ha portato a diversi vantaggi, ad esempio: la riduzione della potenza dissipata, una resistenza maggiore agli effetti delle radiazioni e all'aumento della densita di transistor per unità d'area.  

\vspace*{0.5cm}

L'obbiettivo di questo lavoro di tesi, realizzato presso il laboratorio di microelettronica dell'\textit{Università degli studi di Bergamo}, è quello di osservare come variano i parametri statici (ad esempio tensione di soglia o correnti di perdita $I_{OFF}$) all'aumentare della dose assorbita, in un una tecnologia CMOS a $28nm$. Inoltre si vuole esaminare come essi possono recuperare se non più soggetti a irraggiamenti.     

\vspace*{0.5cm}

Nel capitolo \ref{cap1} verranno introdotte le caratteristiche principali del transistore MOSFET.
Si darà una breve descrizione della struttura dei dispositivi analizzati (sia per i transistori canale N che a canale P) a seguire si introdurranno i diversi parametri statici e le regioni di funzionamento. Verranno poi presentati: il modello di piccolo segnale ed elencate le diverse sorgenti di rumore cui il transistore è soggetto. Per concludere il \hyperref[cap1]{primo} capitolo si esaminerà in che modo le radiazioni possono influire sui parametri statici e come si possono mitigare queste variazioni.

Nel \hyperref[cap2]{secondo} capitolo, oltre ad indicare i dispositivi analizzati e le procedure di estrazione dei dati grezzi, verranno presentati i parametri statici principali, con una analisi più approfondita per la tensione di soglia ($V_{th}$). Per ognuno di essi si descriverà come si possono ricavare e, in seguito, si mostreranno i valori estratti e come variano all'aumentare della dose assorbita.   
