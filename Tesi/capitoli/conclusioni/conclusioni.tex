Questo lavoro di tesi ha avuto come obiettivo la caratterizzazione di una tecnologia CMOS da $28nm$, nello specifico sullo studio della variazione dei parametri statici all'aumentare della dose assorbita.
L'attività condotta nell'ultimo anno, oltre ad effettuare le misurazioni sui diversi MOSFET ad ogni step di irraggiamento (fino a $3Grad$) e dopo l'\textit{annealing}, è consistita nell'implementazione di diverse funzioni MATLab (una piattaforma di programmazione e calcolo numerico) volte ad estrarre i parametri statici presentati.

\vspace{0.5cm}

Gran parte degli sforzi sono stati dedicati all'estrapolazione della tensione di soglia $V_{th}$. Non essendoci un metodo unico per estrarre questo valore si sono utilizzati quattro metodo diversi: \textit{TCM}, \textit{SDLM}, \textit{ERL} e \textit{RM}. Seppur discordanti sul valore della tensione, tali metodi concordano sul comportamento delle curve in funzione dell'irraggimanento subito, ovvero sulla $\Delta V_{th}$ al variare della \textit{TID}. In particolare, nei dispositivi a canale P si ha un incremento continuo del modulo della $V_{th}$, in linea con la teoria. Negli NMOS si è osservato che ad alte dosi il valore della $V_{th}$ diminuisce, ma non si sono trovate spiegazioni per tale comportamento.

L'incremento delle correnti di perdita è molto significativo negli NMOS, in particolare per i dispositivi con alti rapporti $\sfrac{W}{L}$. I PMOS, non soffrendo della creazione di transistor parassiti, non hanno subito cambiamenti significativi, che sono di 2 ordini di grandezza inferiori rispetto ai dispositivi a canale N. Questo parametro è importante in termini di dissipazione di potenza, poiché la corrente di perdita è l'unica presente presente nel MOSFET quando è spento ($I_{OFF}$) e quindi rappresenta l'unica fonte di dissipazione di potenza in tale condizione.   

Sia nei dispositivi a canale N sia in quelli a canale P si è ottenuta una riduzione della transconduttanza $g_m$. Nello specifico, allo step da $3Grad$ si è arrivati ad una riduzione del $10\%$ negli NMOS e ad una del $17\%$ nei PMOS. Al contrario, la reazione all'\textit{annealing} dei due tipi di MOSFET è stata di diversa natura: nei dispositivi a canale P, il trend si è invertito, denotando un recupero parziale della transconduttanza; nei dispositivi a canale N non sono stati osservati cambiamenti rilevanti della $g_m$.    

   