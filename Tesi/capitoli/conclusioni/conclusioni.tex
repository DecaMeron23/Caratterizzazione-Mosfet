Questo lavoro di tesi ha avuto come obbiettivo la caratterizzazione di una tecnologia CMOS da $28nm$, nello specifico sullo studio della variazione dei parametri statici, all'aumentare delle dosi assorbite.
L'attività condotta nell'ultimo anno, oltre ad effettuare le misurazioni sui diversi MOSFET ad ogni step di irraggiamento (fino a $3Grad$) compreso lo step di \textit{annealing}, è stato quello di implementare diverse funzioni MATLab (Piattaforma di programmazione e calcolo numerico) volte ad estrarre i parametri statici presentati.

\vspace{0.5cm}

Gran parte degli sforzi sono stati dedicati all'estrapolazione della tensione di soglia $V_{th}$. Non essendoci un metodo unico per estrarre questo valore si sono utilizzati quattro metodo diversi: \textit{TCM}, \textit{SDLM}, \textit{ERL} e \textit{RM}. I quali, seppur discordanti sul valore della tensione, collimano sull'andamento delle curve, ovvero sulla $\Delta V_{th}$, al variare della \textit{TID}. In particolare nei dispositivi a canale P si ha un incremento continuo del modulo della $V_{th}$, in linea con la teoria. Mentre negli NMOS, per alte dosi, non si sono trovate spiegazioni per gli andamenti delle curve.

L'incremento delle correnti di perdita è stato molto significativo nei NMOS, in particolare per i dispositivi con  alti rapporti $\sfrac{W}{L}$. I PMOS, non soffrendo della creazione di transistor parassiti, non hanno subito cambiamenti significativi, nello specifico 2 ordini di grandezza inferiori rispetto al canale N. Questo parametro è importante in termini di dissipazione di potenza, infatti esse sono le correnti del MOSFET quando è spento ($I_{OFF}$).   

In entrambi in canali si è ottenuta una riduzione della transconduttanza $g_m$. Nello specifico allo step da $3Grad$ si è arrivati fino ad una riduzione del $10\%$ (nei NMOS), mentre nei PMOS fino al $17\%$. Diversamente è stata la reazione all'\textit{annealing}. Infatti, nei dispositivi a canale P, il trend si è invertito denotando un recupero parziale della transconduttanza, mentre nei canali N non si sono verificati cambiamenti importanti della $g_m$.    





%   - Commenti generali sulle variazioni dei parametri     