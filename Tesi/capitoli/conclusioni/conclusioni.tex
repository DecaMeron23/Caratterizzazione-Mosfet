Questo lavoro di tesi ha avuto come obiettivo la caratterizzazione di una tecnologia CMOS da $28nm$, sotto il profilo della sua resistenza alle radiazioni ionizzanti per un possibile impiego in esperimenti di fisica delle alte energie. L'attività condotta nell'ultimo anno, oltre ad effettuare le misure sui diversi MOSFET ad ogni step di irraggiamento (fino a $3Grad$) e dopo l'\textit{annealing}, si è focalizzata nella realizzazione di diverse funzioni MATLab volte ad estrarre i parametri statici e di piccolo segnale presentati in questo lavoro.

\vspace{0.5cm}

Gran parte degli sforzi sono stati dedicati all'estrapolazione della tensione di soglia $V_{th}$. Non essendoci un metodo unico per estrarre questo parametro si sono utilizzati quattro metodo diversi: \textit{TCM}, \textit{SDLM}, \textit{ERL} e \textit{RM}. Questi metodi hanno fornito valori assoluti di tensione di soglia in alcuni casi significativamente differenti, in altri abbastanza simili. Sono stati però tutti in grado di tracciare correttamente la variazione della $V_{th}$ in funzione della dose di radiazioni assorbita. In particolare, nei PMOS la dose assorbita comporta un aumento monotono del valore assoluto della tensione di soglia. Negli NMOS, invece, si è prima osservata una diminuzione della $V_{th}$ e poi un aumento. Questo andamento è dovuto a due effetti differenti e contrapposti: le cariche positive intrappolate negli ossidi di canale e nelle \emph{Shallow Trench Isolation}(\emph{STI}) tendono a ridurre la $V_{th}$, mentre le cariche negative intrappolate all'interfaccia ossido-canale tendono ad aumentare la $V_{th}$.

L'incremento delle correnti di perdita è molto significativo negli NMOS, in particolare per i dispositivi con alti rapporti $\sfrac{W}{L}$. Questo effetto, come già noto in letteratura, è dovuto alla carica positiva intrappolata negli ossidi delle \emph{STI} che creano uno strato di inversione che connette il Drain con il Source anche quando non è presente una tensione $V_{GS}$. I PMOS, non soffrendo della creazione di transistor parassiti, non hanno subito cambiamenti significativi nella corrente di \emph{leakage}. Questo parametro è importante in termini di dissipazione di potenza, poiché la corrente di perdita è l'unica presente nel MOSFET quando è spento ($I_{off}$) e quindi rappresenta l'unica fonte di dissipazione di potenza in tale condizione.   

Sia nei dispositivi a canale N sia in quelli a canale P si è ottenuta una riduzione della transconduttanza $g_m$. Nello specifico, allo step da $3Grad$ si è arrivati ad una riduzione del $10\%$ negli NMOS e ad una del $17\%$ nei PMOS. Al contrario, la reazione all'\textit{annealing} dei due tipi di MOSFET è stata di diversa natura: nei dispositivi a canale P, il trend si è invertito, denotando un recupero parziale della transconduttanza; nei dispositivi a canale N non sono stati osservati cambiamenti rilevanti della $g_m$.    

   