Il guadagno intrinseco $A_{vi}$ è definito come il massimo guadagno ottenibile da un MOSFET polarizzato da un generatore di corrente ideale.
Tale parametro fornisce una misura di quanto un MOSFET può amplificare senza essere influenzato da elementi esterni.
Esso viene calcolato come:

\begin{equation}
    A_{vi} = g_{m} \cdot r_0 = \frac{g_{m}}{g_{ds}}
\end{equation}
Dove  $g_m$ è la transconduttanza e $r_0 = \frac{1}{g_{ds}}$ è la resistenza in uscita dal transistor. Spesso, il grafico di $A_{vi}$ viene mostrato in funzione del coefficiente di inversione $I_{C0}$, che è un parametro utile per descrivere il grado di inversione del canale: debole per valori inferiori a $0.1$, moderata per valori compresi tra $0.1$ e $10$, forte per valori superiori a $10$. Per alti valori di $V_{DS}$, $I_{C0}$ può essere ricavato dalla $I_D$:

\begin{equation}
    I_{C0} = \frac{I_{D}}{{I_{Z}}^{*}} \cdot \frac{L}{W}
\end{equation}

La corrente caratteristica ${I_{Z}}^{*}$ è stata misurata pre-irraggiamento. I valori sono riportati nella tabella \ref{tab:corrente_caratteristica}.

\begin{table}[ht]
    \centering
    \begin{tabular}{c c}
        \toprule
        Tipologia Canale & ${I_{Z}}^{*}[nA]$ \\
        \midrule
        N                & $470$     \\
        P                & $370$     \\
        \bottomrule
    \end{tabular}
    \caption[Valori estratti $I_z^*$]{Valori della corrente caratteristica misurati prima dell'irraggiamento}
    \label{tab:corrente_caratteristica}
\end{table}


A figura \ref{fig:variazione_guadagnoIntrinseco} vengono mostrati i grafici $A_{vi}$ - $I_{C0}$ dei transistori MOSFET a canale N e P raggruppati per larghezza di canale, prima di essere irraggiati e dopo l'irraggiamento a $3Grad$, 
% Mentre nei NMOS si nota un innalzamento significativo della curva $A_{vi}$ - $I_{C0}$, soprattutto per lunghezze di canale più grandi. Questo non sembra verificarsi per i PMOS, infatti, l'offset della curva è leggermente negativo rispetto ai rispettivi grafici pre-irraggiamento. Effetto più visibile per le lunghezze di canale più grandi.
% \todo[inlinepar]{Dare una spiegazione a questo effetto(?)}  


\begin{figure}[ht]
    \centering
    %W = 100
    \includegraphics[width=0.49\textwidth]{./capitolo2/Guadagno_Intrinseco/variazione_guadagno/guadagnoIntrinsecoW100N4.png}
    \includegraphics[width=0.49\textwidth]{./capitolo2/Guadagno_Intrinseco/variazione_guadagno/guadagnoIntrinsecoW100P1.png}\\
    \vspace{0.2cm}
    %W = 200
    \includegraphics[width=0.49\textwidth]{./capitolo2/Guadagno_Intrinseco/variazione_guadagno/guadagnoIntrinsecoW200N4.png}
    \includegraphics[width=0.49\textwidth]{./capitolo2/Guadagno_Intrinseco/variazione_guadagno/guadagnoIntrinsecoW200P1.png}\\
    \vspace{0.2cm}
    %W = 600
    \includegraphics[width=0.49\textwidth]{./capitolo2/Guadagno_Intrinseco/variazione_guadagno/guadagnoIntrinsecoW600N4.png}
    \includegraphics[width=0.49\textwidth]{./capitolo2/Guadagno_Intrinseco/variazione_guadagno/guadagnoIntrinsecoW600P1.png}

    \caption[Variazione del guadagno intrinseco pre e post irraggiamento]{Variazioni del guadagno intrinseco per NMOS, a sinistra, e PMOS a destra prima di essere stati irraggiati e dopo aver subito una dose irraggiamento di $3Grad$.}
    \label{fig:variazione_guadagnoIntrinseco}
\end{figure}