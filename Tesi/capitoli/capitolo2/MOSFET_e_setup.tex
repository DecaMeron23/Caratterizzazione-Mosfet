Per questo studio sono stati utilizzati due chip, uno composto da 9 MOSFET a canale N, l'altro composto da 9 dispositivi a canale P. In entrambi i chip, tutti i dispositivi hanno il Gate spesso $28 nm$ e composto da materiali isolanti con un'alta costante dielettrica (\emph{High-K Metal Gate (HKMG) transistors}), ma avente comunque un sottilissimo strato di biossido di silicio che compone l'interfaccia con il Substrato, che è del tipo $Si/SiO_2$. \\

I dispositivi di ciascun chip hanno dimensioni differenti. In entrambi i chip, dunque, è presente un dispositivo di ogni dimensione presente nella tabella \ref{tab:dimensioniMOSFET}.

\begin{table}[ht]
  \renewcommand{\arraystretch}{1.3}
  \centering
    \begin{tabular}{c c }
      \toprule
          Larghezza $W$ $[\mu m]$ & Lunghezza $L$ $[\mu m]$ \\
     \midrule
	   100 & 0.030 \\
	\hline
	   100 & 0.060 \\
	\hline
	   100 & 0.180 \\
	\hline
	   200 & 0.030 \\
 	\hline
	   200 & 0.060 \\
	\hline
	   200 & 0.180 \\
	\hline
 	   600 & 0.030 \\
	\hline
	   600 & 0.060 \\
	\hline
	   600 & 0.180 \\
      \bottomrule
    \end{tabular}
 
  \caption{Dimensioni dei dispositivi presenti in ciascun chip usato per lo studio}
  \label{tab:dimensioniMOSFET}
\end{table}

Durante la fase di misurazione l'NMOS di dimensioni $600-0.030$ e il PMOS di dimensioni $100-0.180$ sono stati rotti, dunque i dati mostrati e le analisi discusse non li terranno in considerazione.\\

Ogni chip ha subito il processo di irraggiamento diverse volte, presso l'università di Padova, portando il livello di \emph{TID} a valori sempre più alti. Ad ogni step di irraggiamento sono state compiute le misure per estrarre i dati utili per l'analisi dei parametri statici. I valori di \emph{TID} a cui queste sono state compiute sono: $0 rad$ (ovvero pre-irraggiamento, $5 Mrad$, $50 Mrad$, $100 Mrad$, $200 Mrad$, $600 Mrad$, $1 Grad$, $3 Grad$. Infine, i dispositivi hanno subito un processo di \emph{annealing} a $100\degree C$ per $24$ ore.

Per poter analizzare i parametri statici dei MOSFET, ad ogni livello di irraggiamento e dopo il processo di \emph{annealing} sono state misurate le seguenti caratteristiche:

\begin{itemize}

\item{Caratteristica $I_D-V_{DS}$} per diversi valori di $V_{GS}$, ottenuta misurando la corrente di Drain facendo variare i valori delle tensioni come mostrato nella tabella \ref{tab:tensioniIdVds}:

\begin{table}[ht]
  \renewcommand{\arraystretch}{1.3}
  \centering
    \begin{tabular}{c c c c}
      \toprule
         & $V_{min}$ $[mV]$ & $V_{max}$ $[mV]$ & $\Delta V$ $[mV]$ \\
     \midrule
	  $V_{DS}$ & 0 & 900 & 5 \\
	\hline
	   $V_{GS}$ & 0 & 900 & 150\\
      \bottomrule
    \end{tabular}
  \caption[Valori delle tensioni per la misura della caratteristica $I_D-V_{DS}$]{Valori delle tensioni per la misura della caratteristica $I_D-V_{DS}$. Per i PMOS i valori hanno segno opposto rispetto a quello riportato}
  \label{tab:tensioniIdVds}
\end{table}

\item{Caratterstica $I_D-V_{GS}$} per diversi valori di $V_{DS}$, ottenuta misurando la corrente di Drain facendo variare i valori delle tensioni con due modalità, mostrate nella tabella \ref{tab:tensioniIdVgs}. 

\begin{table}[ht]
  \renewcommand{\arraystretch}{1.3}
  \centering
    \begin{tabular}{c c c c c}
      \toprule
        & & $V_{min}$ $[mV]$ & $V_{max}$ $[mV]$ & $\Delta V$ $[mV]$ \\
     \midrule
	  \multirow{2}{*}{Modalità 1} & $V_{DS}$ & 0 & 100 & 10 \\
	\cmidrule{2-5}
	  & $V_{GS}$ & -300 & 900 & 5\\
	\midrule
	  \multirow{2}{*}{Modalità 2} & $V_{DS}$ & 0 & 900 & 150 \\
	\cmidrule{2-5}
	  & $V_{GS}$ & -300 & 900 & 5\\
      \bottomrule
    \end{tabular}
  \caption[Valori delle tensioni per la misura della caratteristica $I_D-V_{GS}$]{Valori delle tensioni per la misura della caratteristica $I_D-V_{GS}$. Per i PMOS i valori hanno segno opposto rispetto a quello riportato}
  \label{tab:tensioniIdVgs}
\end{table}

\end{itemize}


Le misure ottenute sono state analizzate con dei programmi scritti in MatLab. Di seguito si discutono vari parametri estratti.
