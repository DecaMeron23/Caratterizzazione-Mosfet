Per questo studio sono stati utilizzati due chip, uno composto da 9 MOSFET a canale N, l'altro composto da 9 dispositi a canale P. In entrambi i chip, tutti i dispositivi sono in tecnologia $28 nm$ e hanno dimensioni differenti. In entrambi i chip, dunque, è presente un dispositivo di ogni dimensione presente nella tabella \ref{tab:dimensioniMOSFET}.

\begin{table}[H]
  \renewcommand{\arraystretch}{1.3}
  \centering
    \begin{tabular}{c c }
      \toprule \\
          Larghezza $W$ $[\mu m]$ & Lunghezza $L$ $[\mu m]$ \\
     \midrule
	   100 & 0.030 \\
	\hline
	   100 & 0.060 \\
	\hline
	   100 & 0.180 \\
	\hline
	   200 & 0.030 \\
 	\hline
	   200 & 0.060 \\
	\hline
	   200 & 0.180 \\
	\hline
 	   600 & 0.030 \\
	\hline
	   600 & 0.060 \\
	\hline
	   600 & 0.180 \\
      \bottomrule
    \end{tabular}
 
  \caption{Dimensioni dei dispositivi presenti in ciascun chip usato per lo studio}
  \label{tab:dimensioniMOSFET}
\end{table}

Durante la fase di misurazione, però, l'NMOS di dimensioni $600-0.030$ e il PMOS di dimensioni $100-0.180$ sono stati rotti, dunque i dati mostrati e le analisi discusse non li terranno in considerazione.

\todo[inline]{Descrizione della board di misurazione e delle tensioni utilizzate per le misure statiche}
