Il guadagno intrinseco ($A_{vi}$) è definito come il massimo guadagno ottenibile da un MOSFET, polarizzato da un generatore di corrente ideale; esso viene calcolato come:

$$A_{vi} = g_{m} \cdot r_0 = \frac{g_{m}}{g_{ds}} $$

Con $g_m$ la transconduttanza e $r_0$ la resistenza in uscita dal transistor (${1}/{g_{ds}}$). Spesso, il grafico di $A_{vi}$, viene mostrato in funzione del coefficente di inversione ($I_{C0}$), esso si può ricavare dalla $I_d$ ad alte $V_{ds}$:

$$I_{C0} = I_{d} \cdot I_{z}^{*} \cdot \frac{L}{W}$$

Dove la corrente caratteristica, $I_{z}^{*}$, è stata misurata pre-irraggiamento e i valori sono riportati alla tabella \ref{tab:corrente_caratteristica}.

\begin{table}[h!]
    \centering
    \begin{tabular}{c c}
        \toprule
        Tipologia Canale & $I_{z}^{*}$ \\
        \midrule
        N                & $470nA$    \\
        P                & $370nA$    \\
        \bottomrule
    \end{tabular}
    \caption{Valori della corrente caratteristica misurati prima dell'irraggiamento}
    \label{tab:corrente_caratteristica}
\end{table}
