
Gli effetti principali delle radiazioni sui dispositivi elettronici sono di due categorie:\cite{bib:Effetti_Radiazioni_1987}:
\begin{itemize}
	\item Ionizzazione, generazione di coppie elettroni-lacune($e-h$)
	\item Danni da spostamento, dislocazione degli atomi dai loro siti reticolari.
\end{itemize}
Mentre il danno da spostamento (DD) non è molto rilevante sui MOSFET, la ionizzazione può comportare variazioni dei parametri elettrici come guadagno e tensione di soglia, per questo motivo ci concentreremo solo su gli effetti della ionizzazione.
\\

Il passaggio di una particella ionizzante, all'interno della materia, comporta una perdita di energia. Questo dissipamento può essere formalizzato come:

$$ \Delta E = \Delta E_{\text{elettronica}} + \Delta E_{nucleare} $$

Perdita di energia elettronica e perdita di energia nucleare, la prima è dovuta dalle iterazioni con gli elettroni negli atomi, mentre la seconda è causata dalle iterazioni con i nuclei degli atomi.
Gran parte dell'energia persa è elettronica, occasionalmente avvengono \textit{collisioni forti} (hard collision) tali da creare frammenti nucleari, pertanto la perdita di energia totale è una buona approssimazione della perdita dell'energia elettronica\cite{bib:Effetti_Radiazioni_NASA}.
\\

\subsection{Ionizzazione}
La perdita di energia elettronica nei semiconduttori provoca la ionizzazione, creazione di coppie elettrone-lacuna ($e-h$). Il numero di $e-h$ è proporzionale all'energia elettronica persa da parte della particella (per il slicio la constante è di 1/3,6eV mentre per SiO2 è di 1/18eV \cite{bib:Effetti_Radiazioni_NASA}).
\\

L'effetto di ionizzazione è diverso nei semiconduttori e nei isolanti, una differenza è che l'effetto si accumuli o si dissipi rapidamente.
Negli isolanti, come l'ossido di gate nei MOS, è presente l'effetto cumulativo. Gli elettroni liberati da una particella ionizzante si possono muovere più facilmente soprattutto grazie a effetti di campo dovuti, ad esempio, a polarizzazioni.
Al contrario i buchi(holes) sono molto meno mobili, la maggior parte di essi sopravvivono alla ricombinazione con gli elettroni. Creando perciò una carica positiva all'interno dell'ossido, con uno degli effetti di spostare la tensione di soglia nei MOSFET.
\\

La ricottura dell'ossido (\textit{oxide anneals}) è una procedura volta a ripristinare, almeno in parte, le caratteristiche del MOSFET a seguito di dose subita.
Consiste nel neutralizzare i buchi scaldando il dispositivo, ad esempio, la procedura di annealing che abbiamo eseguito sui nostri dispositivi è stata di portarli ad una temperatura di $100\degree C$ e mantenerla per ventiquattro ore. La ricottura può avvenire anche a temperatura ambiente, ma probabilmente i tempi sarebbero molto lunghi, anche anni\cite{bib:Effetti_Radiazioni_NASA}. 


\subsection{Misurazione Dose subita}
Per effettuare la misurazione della ionizzazione accumulata si utilizza somma dell'energia persa, da parte delle particelle ionizzanti, sul materiale. Quindi in buona approssimazione, l'energia totale creata dalle interazioni con gli elettroni degli atomi alla materia e le particelle, è una misura della dose subita\cite{bib:Effetti_Radiazioni_NASA}. Essa viene chiamata TID (\textit{total ionizing dose}) ed è espressa in Energia su unità di massa.
La TDI normalmente è misurata in $rad$\footnote{radiation absorbed dose}, definito come 100 $erg$\footnote{Unità di lavoro corrispondente a $10^{-7} J$} per grammo di materiale, oppure, accolto dal sistema internazionale, il gray $Gy$ dove $Gy = 100rad = 1\frac{J}{Kg}$.


% \subsection{Danno da spostamento}
% \todo[inlinepar]{Non ha effetti sui MOSFET}
% Quando una particella (neutrone, protone o anche elettrone ad alta energia) possiede abbastanza energia tale da poter dislocare un atomo al di fuori dalla sua posizione normale all'interno reticolo, in un semiconduttore, avviene il danno da spostamento (\textit{Displacement damage}), in particolare dal movimento dell'atomo all'interno del semiconduttore. Può verificarsi che l'atomo colpito ne colpisa altri, a patto che abbia abbsatanza energia, a sua volta\cite{bib:Effetti_Radiazioni_1987}.

% Il danno da spostamento dipende dall'energia della particella, in particolare i protoni, avendo una massa maggiore, consentono di trasferire all'atomo colpito una energia maggiore rispetto ai elettroni\cite{bib:Effetti_Radiazioni_su_dispositivi_optoeltronici}.