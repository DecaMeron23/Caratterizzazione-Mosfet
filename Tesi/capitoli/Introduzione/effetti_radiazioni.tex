
Gli effetti delle radiazioni sui MOSFET possono essere principalmente di due categorie\cite{bib:Effetti_Radiazioni_1987}:
\begin{itemize}
	\item Ionizzazione, generazione di coppie elettroni-lacune($e-h$)
	\item Danni da spostamento, dislocazione degli atomi dai loro siti reticolari.
\end{itemize}
Il passaggio di una particella ionizzante, all'interno della materia, comporta una perdita di energia. Questo dissipamentdo può essere formalizzato come:

$$ \Delta E = \Delta E_{\text{elettronica}} + \Delta E_{nucleare} $$

Perdita di energia elettronica e perdita di energia nucleare, la prima è dovuta dalle iterazioni con gli elettroni negli atomi, mentre la seconda è causata dalle iterazioni con i nuclei degli atomi.
Gran parte dell'energia persa è elettronica, occasionalmente avvengono \textit{collisioni forti} (hard collision) tali da creare frammenti nucleari, pertanto la perdita di energia totale è una buona approssimazione della perdita dell'energia elettronica\cite{bib:Effetti_Radiazioni_NASA}.
\\

\subsection{Ionizzazione}
La perdita di energia elettronica nei semiconduttori provoca la ionizzazione, creazione di coppie elettrone-lacuna ($e-h$). Il numero di $e-h$ è proporzionale all'energia elettronica persa da parte della particella (per il slicio la constante è di 1/3,6eV mentre per SiO2 è di 1/18eV \cite{bib:Effetti_Radiazioni_NASA}).
\\

L'effetto di ionizzazione è diverso nei semiconduttori e nei isolanti, una differenza è che l'effetto si accumuli o si dissipi rapidamente.
Negli isolanti, come l'ossido di gate nei MOS, è presente l'effetto cumulativo. Gli elettroni liberati da una particella ionizzante si possono muovere più facilmente soprattutto grazie a effetti di campo dovuti, ad esempio, a polarizzazioni.
Al contrario i buchi(holes) sono molto meno mobili, la maggior parte di essi sopravvivono alla ricombinazione con gli elettroni. Creando perciò una carica positiva all'interno dell'ossido, con uno degli effetti di spostare la tensione di soglia nei MOSFET.
\\
I buchi si possono neutralizzare tramite la ricottura dell'ossido (oxide anneals) in tempi molto lunghi a temperatura ambiente


\subsubsection*{Misurazione Dose subita}
La carica accumulata intrappolate viene misurata dalla ionizzazione accumulata, che a sua volta viene misurata dalla somma dell'energia persa dalle particelle sul materiale attraverso le interazioni con gli elettroni (\textit{which in turn is measured by the sum(over particles) of the energy lost by the perticles to the material via interactions with the electrons}).

Quindi una misura utile è l'energia totale per unità di massa di materiale trasferita dalle particelle ionizzanti al materiale attraverso ionizzazione, essa viene chiamata TID (total ionizing dose).
La TDI normalmente è misurata in rad\footnote{radiation absorbed dose} quale è definito come 100 erg\footnote{Unità di lavoro corrispondente a $10^{-7}$ di energia depositata per grammo di materiale}.


\subsection{Danno da spostamento}
\todo[inlinepar]{Forse non ha effetti sui MOSFET}
Quando una particella (neutrone, protone o anche elettrone ad alta energia) possiede abbastanza energia tale da poter dislocare un atomo al di fuori dalla sua posizione normale all'interno reticolo, in un semiconduttore, avviene il danno da spostamento (\textit{Displacement damage}), in particolare dal movimento dell'atomo all'interno del semiconduttore, inoltre può anche verificarsi che l'atomo colpito ne colpisa altri a sua volta\cite{bib:Effetti_Radiazioni_1987}.

Il danno da spostamento dipende dall'energia della particella, in particolare i protoni, avendo una massa maggiore, consentono di trasferire all'atomo colpito una energia maggiore rispetto ai elettroni\cite{bib:Effetti_Radiazioni_su_dispositivi_optoeltronici}.