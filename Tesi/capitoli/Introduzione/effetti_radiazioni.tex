\todo[inline]{
    Effetti di radiazioni possono essere di tre categorie Ionizzazione cumulativa, effetti di singoli eventi e danni da spostamento [JPL00-62.pdf]

La perdita di energia di una particella ionizzante mentre viaggia attraverso la materia può essere divisa in due componenti: perdita di energia elettroniche e perdita di energia nucleare, la prima è dovuta dalle iterazioni con gli elettroni negli atomi e la seconda è causata dalle iterazioni con i nuclei degli atomi. Gran parte dell'energia persa è elettronica, occasionalmente avvengono "collisioni dure" (hard collisions) tali da creare frammenti nucleari, pertanto la perdita di energia totale è una buona approssimazione della perdita dell'energia elettronica.

La perdita di energia elettronica nei semiconduttori provoca la ionizzazione, creazione di coppie elettrone-lacuna (e-h). Il numero di e-h è proporzionale all'energia elettronica persa da parte della particella (per il slicio la constante è di 1/3,6eV mentre per SiO2 è di 1/18eV)

(Non faccio la parte dei fotoni)

l'effetto di ionizzazione e diverso nei semiconduttori e nei isolanti, una differenza è che l'effetto si accumuli o si dissipi rapidamente

negli isolanti, come l'ossido di gate nei MOS, è presente l'effetto cumulativo




Due effetti principali: [ADA186936.pdf]
- Ionizzazione (generazione di coppie elettrone/foro)
	Ionization(generation of electron/hole pairs)
- Danni da spostamento(dislocare gli atomi dai loro siti reticolari normali)
	displacement damage (dislodging atoms from their normal lattice sites)
}