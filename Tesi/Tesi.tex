\documentclass[
	a4paper,
	cleardoublepage=empty,
	headings=twolinechapter,
	numbers=autoenddot,
]{scrbook}

%% Vecchia template
%\documentclass[12pt, letterpaper]{book}
\usepackage{graphicx} %LaTeX package to import graphics
\graphicspath{{./Immagini}} %configuring the graphicx package

\usepackage[T1]{fontenc}
\usepackage[italian]{babel}

\usepackage{hyphenat}
\usepackage{array}
\usepackage{booktabs} % Per linee orizzontali migliori
\usepackage{caption}  % Per personalizzare le didascalie<
\usepackage{multirow} % Per combinare celle nelle colonne
\usepackage{float}
\usepackage{hyperref}

\usepackage{todonotes} % mettere le note dentro il documento

\usepackage{siunitx}  % Per formattare le unità di misura
\usepackage{gensymb} % Simboli come °

\usepackage{import}
\usepackage{frontespizio}

\begin{document}
% Inizio introduzione documento
\frontmatter

%frontespizio
\import{./frontespizio}{frontespizio.tex}

% Indice
\tableofcontents
% Lista delle figure
% \listoffigures

% Introduzione della tesi
\chapter*{Abstract}



% Inizio Parte principale documento
\mainmatter

\chapter{Introduzione}
\todo[inlinepar]{Cambiare il titolo del capitolo(?)}
In questo capitolo si cercano di dare le informazioni principali sugli strumenti utilizzati e sulle cause degli effetti che analizzeremo nei capitoli successivi.
\todo[inline]{Fare una introduzione al capitolo}
\section{Effetti delle radiazioni sui MOSFET}
\import{./capitoli/Introduzione}{effetti_radiazioni.tex}


%Capitolo: Estrazione dei parametri statici
\chapter{Estrazione dei parametri statici}
In questo capitolo si tratteranno i parametri statici di transistori \emph{MOS} in tecnologia $28 nm$ per comprendere come variano le prestazioni statiche all'aumentare dell'irraggiamento subito. Si tratteranno:\todo{Aggiungere i parametri statici che studieremo}
\begin{itemize}
  \item Tensione di soglia
  \item Transconduttanza
  \item $G_{ds}$
  \item $I_{on}$
  \item $I_{off}$
  \item Guadagno Intrinseco
\end{itemize}
\section{Tensione di soglia}
\import{./capitoli/Misure_Statiche}{tensione_di_soglia.tex}

\section[$G_{m}$]{Transconduttanza}

\section[$G_{ds}$]{$G_{ds}$}

\section[$I_{on}$]{$I_{on}$}
\section[$I_{off}$]{$I_{off}$}
\section[Guadagno Intrinseco]{Guadagno Intrinseco}
\import{./capitoli/Misure_Statiche}{guadagno_intrinseco.tex}

% Inizio Parte finale
\backmatter

\bibliographystyle{IEEEtrans}
\bibliography{./bibliografia/bibliography.bib}

\end{document}