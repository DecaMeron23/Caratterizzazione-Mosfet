\documentclass[
	a4paper,
	cleardoublepage=empty,
	headings=twolinechapter,
	numbers=autoenddot,
]{scrbook}

%% Vecchia template
%\documentclass[12pt, letterpaper]{book}
\usepackage{graphicx} %LaTeX package to import graphics
\graphicspath{{./Immagini}} %configuring the graphicx package

\usepackage[T1]{fontenc}
\usepackage[italian]{babel}

\usepackage{hyphenat}
\usepackage{array}
\usepackage{booktabs} % Per linee orizzontali migliori
\usepackage{caption}  % Per personalizzare le didascalie<
\usepackage{multirow} % Per combinare celle nelle colonne
\usepackage{float}
\usepackage{hyperref}

\usepackage{todonotes} % mettere le note dentro il documento

\usepackage{siunitx}  % Per formattare le unità di misura
\usepackage{gensymb} % Simboli come °
\usepackage{xfrac} %per fare le frazioni inclinate

\usepackage{subfig} %per fare più figure in uno


\usepackage{import}
\usepackage{frontespizio}

\begin{document}
% Inizio introduzione documento
\frontmatter

%frontespizio
\import{./frontespizio}{frontespizio.tex}

% Indice
\tableofcontents
% Lista delle figure
\listoffigures

% Inizio Parte principale documento
\mainmatter

% Introduzione della tesi
\chapter*{Introduzione}
\import{./capitoli/introduzione}{introduzione.tex}



% ? Il transistore MOSFET
\chapter{Il transistore MOSFET}
\todo[inlinepar]{titolo temporaneo(?)}
\todo[inline]{Fare una introduzione al capitolo}
In questo capitolo si daranno le principali informazioni necessarie per i capitoli successivi.
% ! MOSFET
\section{Transistore MOSFET}
\import{./capitoli/capitolo1}{transistore_MOSFET.tex}

% !Effetti sulle Radiazioni
\section{Effetti delle radiazioni sui transistori MOSFET}
\import{./capitoli/capitolo1}{effetti_radiazioni.tex}


% ? Studio sperimentale
\chapter{Studio sperimentale}
In questo capitolo si tratteranno i parametri statici di transistori \emph{MOS} in tecnologia $28 nm$ per comprendere come variano le prestazioni statiche all'aumentare dell'irraggiamento subito. Si tratteranno:\todo{Aggiungere i parametri statici che studieremo}
\begin{itemize}
  \item Tensione di soglia
  \item Transconduttanza
  \item Corrente di leakage
  \item Corrente $I_{on}$
  \item Guadagno Intrinseco
\end{itemize}

% ! Tensione di soglia
\section{Variazione della tensione di soglia}
\label{cap2:vth}
\import{./capitoli/capitolo2}{tensione_di_soglia.tex}

% ! Transconduttanza
\section{Variazione della transconduttanza}
\import{./capitoli/capitolo2}{transconduttanza.tex}

% ! I_off
\section{Variazione della corrente di leakage}
\import{./capitoli/capitolo2}{i_off.tex}

% ! I_on
\section{Variazione della corrente $I_{on}$}
\import{./capitoli/capitolo2}{i_on.tex}

% ! Guadagno intrinseco
\section{Guadagno Intrinseco}
\import{./capitoli/capitolo2}{guadagno_intrinseco.tex}


% ? Conclusioni
\chapter*{Conclusioni}
\import{./capitoli/conclusioni}{conclusioni.tex}

% Inizio Parte finale
\backmatter


\bibliographystyle{IEEEtrans}
\bibliography{./bibliografia/bibliography.bib}

\end{document}